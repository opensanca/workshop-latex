%
% citacoes-versos.tex
%
% Rafael Beraldo <rberaldo@cabaladada.org>
% Workshop de LaTeX do Opensanca
%
% Demonstra:
% - Os ambientes quote, quotation e verse
%

\documentclass[a4paper,oneside]{article}
\usepackage{fontspec}
\usepackage{polyglossia}
  \setdefaultlanguage{brazil}
\usepackage{microtype}

\begin{document}
\frenchspacing

\section{Citações}

Os ambientes \texttt{quote} e \texttt{quotation} são úteis para citações. O
primeiro é mais adequado para citações curtas, enquanto que o segundo funciona
bem para citações longas, pois os parágrafos são indentados.

\begin{quote}
  Não entre em pânico!\hfill (Douglas Adams)
\end{quote}

\section{Poemas}

O ambiente \texttt{verse} é particularmente adequado para poemas, pois quebra
as linhas de maneira a indicar quebras intencionais e aquelas causadas pelo
tamanho da página.

% Usar \parbox{5cm}{poema} para demonstrar a quebra de linhas quando os versos
% do poema são longos demais.
\begin{verse}
  O vinho dá-te o calor que não tens;\\
  suaviza o jugo do passado e te alivia\\
  das brumas do futuro; inunda-te de luz\\
  e te liberta desta prisão.
  \flushright
  (Omar Khayyam)
\end{verse}
\end{document}
