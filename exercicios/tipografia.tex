%
% tipografia.tex
%
% Rafael Beraldo <rberaldo@cabaladada.org>
% Workshop de LaTeX do Opensanca
%
% Demonstra:
% - Diferença entre traços horizontais como o travessão e o hífen
% - Como usar aspas corretamente
% - Reticências versus pontos finais
% - Espaços duros
% - Textos multilínguas
% - Macros
%

\documentclass[a4paper,oneside]{article}
\usepackage{fontspec}
\usepackage{polyglossia}
  \setdefaultlanguage{brazil}
  \setotherlanguage{english}
\usepackage{microtype,minted}

% Definindo um novo comando
\newcommand{\filename}[1]{\texttt{#1}}

\begin{document}
\frenchspacing

\section{Tipografia}

Travessões --- os traços horizontais mais longos que vivem no meio de sentenças
--- são muito diferentes das meias-riscas e hifens, por exemplo. Meias-riscas
são utilizadas para juntar dois extremos, como as páginas 20--45 e a viagem de
São Carlos--São Paulo. Não se esqueça de levar o guarda-chuva quando for à
terra da garoa.

Aspas também merecem nossa "atenção". A ironia vem de usar aspas como estas, ao
invés de usar as aspas “corretas”. Também é possível fazê-las ``assim''.

Reticências podem vir do mundo Unicode… Ou de um comando do \LaTeX\ldots\ Mas
nunca de três pontos finais juntos...

Finalmente, é importante lembrar que:

\begin{quote}
  Não se esqueça de adicionar espaços duros quando falar da página 10, da
  Sra. Joana e do Dom João, também.
\end{quote}

\begin{quote}
  Não se esqueça de adicionar espaços duros quando falar da página~10, da
  Sra.~Joana e do Dom~João, também.
\end{quote}

Bem melhor, agora.

\section{Textos multilínguas}

Hoje é~\today, mas em inglês é \textenglish{\today}.

A hifenização de um trecho longo em língua estrangeira ficaria toda errada sem
uma ajuda do polyglossia. Basta usar o ambiente \verb+english+, ou qualquer que
seja o nome da língua.

\section{Macros}

Finalmente, é possível criar macros no \LaTeX. Assim, fica mais fácil se
referir ao arquivo \filename{slides.tex}.

\section{Código-fonte}

É possível incluir código-fonte em seu arquivo usando o pacote \texttt{minted}.

\begin{minted}[autogobble,breaklines]{c}
  #include <stdio.h>

  int main(void)
  {
    printf("Hello, world!\n");
  }
\end{minted}

\end{document}
