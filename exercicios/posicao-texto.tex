%
% posicao-texto.tex
%
% Rafael Beraldo <rberaldo@cabaladada.org>
% Workshop de LaTeX do Opensanca
%
% Demonstra:
% - Ambientes center, flushleft e flushright
% - Como controlar o espaço horizontal e vertical
% - Diferentes unidades que o LaTeX conhece
%

\documentclass[a4paper,oneside]{article}
\usepackage{fontspec}
\usepackage{polyglossia}
  \setdefaultlanguage{brazil}

\begin{document}
\frenchspacing

\section{Ambientes}

Para posicionar o texto horizontalmente na página, podemos usar três
\textbf{ambientes}: \texttt{center, flushleft} e \texttt{flushright}.

\begin{center}
  Todo este texto será centralizado na página.
\end{center}

\begin{flushleft}
  Esse parágrafo será alinhado à esquerda e não será justificado. Mais abaixo,
  veremos como ajustar a posição de um texto dentro da própria linha.
\end{flushleft}

\begin{flushright}
  Também podemos alinhar texto à direita. Além desses ambientes, podemos mudar
  o alinhamento do texto usando os comandos \verb+centering+, \verb+flushleft+
  e \verb+flushright+.
\end{flushright}

\section{Controlando o espaço na linha}

Os comandos \verb+\hspace{comprimento}+ e \verb+\hfill+ nos permitem controlar
o espaço dentro de uma linha:

Aqui haverá\hspace{1.5cm} um espaço.

Começo\hfill meio\hfill fim.

\section{Espaço vertical}

Os comandos para controle do espaço vertical, isto é, o espaço entre os
parágrafos, são muito parecidos com os comandos de espaço horizontal. Eles são
\verb+\vspace{comprimento}+ e \verb+\vfill+.
\end{document}
