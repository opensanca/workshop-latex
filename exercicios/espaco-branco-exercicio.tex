%
% espaco-branco-exercicio.tex
%
% Rafael Beraldo <rberaldo@cabaladada.org>
% Workshop de LaTeX do Opensanca
% 28 de maio de 2016
% 
% Problema: o espaçamento deste texto está catastrófico. Existem espaços
% sobrando ou faltando, tanto entre as palavras quanto entre os parágrafos. O
% texto final deve conter quatro parágrafos, sendo o segundo uma lista cujos
% itens são separados por novas linhas. Corrija os problemas e compile o
% documento.
%

\documentclass{article}
\begin{document}
Acima de tudo, é fundamental ressaltar que o julgamento imparcial das eventualidades maximiza as possibilidades por conta das direções preferenciais no sentido do progresso. Desta maneira,a complexidade dos estudos efetuados cumpre um papel essencial na formulação das condições financeiras e administrativas                       exigidas, que são três:
1. O mundo atual.
2. Os amigos de família.
3. A importância dos mercados mundiais.
Nunca é demais lembrar o peso e o significado destes problemas, uma vez que o fenômeno da Internet exige a precisão e a definição do processo de comunicação como um todo. No entanto, não podemos esquecer que                    a consulta aos diversos militantes agrega valor ao estabelecimento das regras de conduta normativas.Do mesmo modo, o acompanhamento das preferências de consumo garante a contribuição de um grupo importante na determinação das novas proposições.
O empenho em analisar o desenvolvimento contínuo de distintas formas de atuação aponta para a melhoria do sistema de participação geral.  Neste sentido, a competitividade nas transações comerciais desafia a capacidade de equalização dos relacionamentos verticais entre as hierarquias. É importante questionar o quanto o início da atividade geral de formação de atitudes facilita a criação das diretrizes de desenvolvimento para o futuro. Evidentemente, a execução dos pontos do programa promove a alavancagem do retorno esperado a longo prazo.
\end{document}
