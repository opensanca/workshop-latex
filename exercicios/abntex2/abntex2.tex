%
% abntex2.tex
%
% Rafael Beraldo <rberaldo@cabaladada.org>
% Workshop de LaTeX do Opensanca
% 28 de maio de 2016
% 
% Demonstra:
% - Como utilizar a classe abnTeX2 para criar uma tese
% - Organização do texto
% - Comandos específicos da classe
% - O comando \input
%

\documentclass[12pt,oneside,a4paper,brazil]{abntex2}
% Não carregaremos os pacotes polyglossia ou fontspec, pois são carregados por
% padrão pela classe
\usepackage{microtype,graphicx,hyperref,blindtext}
% Pacote usado para citações:
\usepackage[alf]{abntex2cite}


% Informações da capa
\titulo{}
\autor{}
\orientador[Orientadora: ]{}
\data{}
\instituicao{}
\tipotrabalho{}
\preambulo{}

\begin{document}
\selectlanguage{brazil}
\frenchspacing

\pretextual
\imprimircapa
\imprimirfolhaderosto

% Para que a ToC seja incluída nas bookmarks do PDF
\tableofcontents*
\cleardoublepage

\textual
\cleardoublepage
\include{introducao}
\include{objetivos-justificativa}
\include{referencial-teorico}
\include{metodologia}

\postextual
\bibliography{abntex2}
\end{document}
