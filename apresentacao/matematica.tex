% matematica.tex %%%%%%%%%%%%%%
\begin{frame}[standout]
  \Huge
  \filename{matematica.tex}
\end{frame}

% Até agora, estávamos no modo de texto. No modo de matemática, o modo como o
% LaTeX interpreta o que escrever é diferente. Além disso, o modo de matemática
% é dividido em dois tipos: inline e displayed
\begin{frame}
  \frametitle{\filename{matematica.tex}}
  \Huge
  \only<1>{Modo de texto vs.\\ modo de matemática}
  \only<2>{Modo de matemática: \emph{inline} e \emph{displayed}}
\end{frame}

% Existem três ambientes para acessar o modo de matemática.
\begin{frame}[fragile]
  \frametitle{\filename{matematica.tex}}
  \huge
  Três ambientes:\\

  \only<1>{\mintinline{latex}{math} ou \mintinline{latex}{\( … \)}}
  \only<2>{\mintinline{latex}{displaymath} ou \mintinline{latex}{\[ … \]}}
  \only<3>{\mintinline{latex}{equation}}
\end{frame}

% Há uma infinidade de comandos, pacotes e técnicas para aprender. Cobriremos o
% básico.
\begin{frame}
  \frametitle{\filename{matematica.tex}}
  \Huge
  Cobriremos o básico!

  \huge
  Mais em \url{en.wikibooks.org/wiki/LaTEX/Mathematics}
\end{frame}

% Símbolos em modo matemático
\begin{frame}[fragile]
  \frametitle{\filename{matematica.tex}}
  \huge
  \mintinline{latex}{2 \times 2 = 4}

  \[ 2 \times 2 = 4 \]
\end{frame}

% Letras gregas
\begin{frame}[fragile]
  \frametitle{\filename{matematica.tex}}
  \huge
  \mintinline{latex}{\alpha, \beta, \pi}

  \[ \alpha, \beta, \pi \]
\end{frame}

% Operadores
\begin{frame}[fragile]
  \frametitle{\filename{matematica.tex}}
  \begin{minted}[autogobble,fontsize=\Large,breaklines]{latex}
    \cos (2\theta) = \cos^2 \theta - \sin^2 \theta
  \end{minted}

  \huge
  \[ \cos (2\theta) = \cos^2 \theta - \sin^2 \theta \]
\end{frame}

% Potências e subscritos
\begin{frame}[fragile]
  \frametitle{\filename{matematica.tex}}
  \Large

  \begin{minipage}{.65\textwidth}
    \begin{minted}[autogobble,fontsize=\Large,breaklines]{latex}
      2^8
      a_b
      2^{32}
      f(n) = 4n + n^2
    \end{minted}
  \end{minipage}
  \begin{minipage}{.25\textwidth}
    \[ 2^8 \]

    \[ a_b \]

    \[ 2^{32} \]

    \[ f(n) = 4n + n^2 \]
  \end{minipage}
\end{frame}

% Frações
\begin{frame}[fragile]
  \frametitle{\filename{matematica.tex}}
  \Large
  \begin{minted}[autogobble,fontsize=\large,breaklines]{latex}
    F = G \frac{m_1 m_2}{d^2}
    \frac{\frac{1}{x}+\frac{1}{y}}{y-z}
  \end{minted}

  \[ F = G \frac{m_1 m_2}{d^2} \]

  \[ \frac{\frac{1}{x}+\frac{1}{y}}{y-z} \]
\end{frame}

% Raízes
\begin{frame}[fragile]
  \frametitle{\filename{matematica.tex}}
  \Large
  \begin{minipage}{.55\textwidth}
    \begin{minted}[autogobble,fontsize=\Large,breaklines]{latex}
      \sqrt{10^2} = 10
      \sqrt[3]{\frac{a}{b}}
    \end{minted}
  \end{minipage}
  \begin{minipage}{.35\textwidth}
    \[ \sqrt{10^2} = 10 \]

    \[ \sqrt[3]{\frac{a}{b}} \]
  \end{minipage}
\end{frame}

% Estudar mais exemplos de matemática
\begin{frame}
  \frametitle{\filename{matematica.tex}}
  \Huge
  Estudar \filename{matematica.tex}
\end{frame}

% Exercício: reproduza essa equação em matematica-exercicio.tex
\begin{frame}
  \frametitle{\filename{matematica.tex}}
  \Large
  Reproduza em \filename{matematica-exercicio.tex}:

  \huge
  \begin{equation}
    x = \frac{-b \pm \sqrt{b^2 - 4ac}}{2a}
  \end{equation}
\end{frame}
