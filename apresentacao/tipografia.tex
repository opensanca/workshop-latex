% tipografia.tex %%%%%%%%%%%%%%
\begin{frame}[standout]
  \Huge
  \filename{tipografia.tex}
\end{frame}

% Um dos principais motivos para se usar o LaTeX é justamente sua qualidade
% tipográfica.
\begin{frame}
  \frametitle{\filename{tipografia.tex}}
  \Huge
  \begin{quote}
    A tipografia que tem algo a dizer aspira a ser uma espécie de estátua
    transparente.\hfill(Robert Bringhurst)
  \end{quote}
\end{frame}

% Pontuação: quatro traços diferentes
\begin{frame}
  \frametitle{\filename{tipografia.tex}}
  \Huge
  \begin{itemize}
    \item O travessão: —
    \item A meia-risca: –
    \item O hífen: -
    \item O sinal de menos: \( - \)
  \end{itemize}
\end{frame}

% O travessão
\begin{frame}[fragile]
  \frametitle{\filename{tipografia.tex}}
  \LARGE
  Travessão: —

  \mintinline{latex}{--- Como assim? --- Ela disse.}

  --- Como assim? --- Ela disse.
\end{frame}

% A meia-risca
\begin{frame}[fragile]
  \frametitle{\filename{tipografia.tex}}
  \LARGE
  Meia-risca: –

  \mintinline{latex}{páginas 10--15}

  páginas 10--15
\end{frame}

% O hífen
\begin{frame}[fragile]
  \frametitle{\filename{tipografia.tex}}
  \LARGE
  Hífen: -

  \mintinline{latex}{guarda-chuva}

  guarda-chuva
\end{frame}

% Sinal de menos
\begin{frame}[fragile]
  \frametitle{\filename{tipografia.tex}}
  \LARGE
  Sinal de menos: \( - \)

  \mintinline{latex}{\( 15 - 15 \)}

  \( 15 - 15 \)
\end{frame}

% Aspas: diferença entre aspas retas e curvas
\begin{frame}
  \frametitle{\filename{tipografia.tex}}
  \huge
  \begin{itemize}
    \item Aspas retas: "Olá, mundo".
    \item Aspas curvas: “Olá, mundo".
  \end{itemize}
\end{frame}

% Sintaxe: `` e '' ou “”
\begin{frame}[fragile]
  \frametitle{\filename{tipografia.tex}}
  \Huge
  \mintinline{latex}{``Olá, mundo''.}
  \mintinline{latex}{“Olá, mundo”.}
  \vspace{1em}

  ``Olá, mundo''.\\
  “Olá, mundo”.
\end{frame}

% Reticências
\begin{frame}[fragile]
  \frametitle{\filename{tipografia.tex}}
  \Huge
  \mintinline{latex}{Olá, mundo\ldots}

  Olá, mundo\ldots
\end{frame}

% Espaços duros
\begin{frame}[fragile]
  \frametitle{\filename{tipografia.tex}}
  \LARGE
  Espaços duros: \verb+~+\\

  % Sem espaços duros
  \only<1>{
    \begin{center}
      \fbox{
        \begin{minipage}{.5\textwidth}
          O Sr. Roberto disse que a Sra. Eduarda quer 100 páginas até o dia 5.
          Às 9 eu só havia terminado 60!
        \end{minipage}
      }
    \end{center}
  }

  % Com espaços duros
  \only<2>{
    \begin{center}
      \fbox{
        \begin{minipage}{.5\textwidth}
          O Sr.~Roberto disse que a Sra.~Eduarda quer~100 páginas até o dia~5.
          Às~9 eu só havia terminado~60!
        \end{minipage}
      }
    \end{center}
  }
\end{frame}

% A classe memoir vem com várias melhorias por padrão.
\begin{frame}
  \frametitle{\filename{tipografia.tex}}
  \Huge
  Recomendação: a classe \code{memoir}
\end{frame}

% Textos bilíngues
\begin{frame}
  \frametitle{\filename{tipografia.tex}}
  \Huge
  Textos com várias línguas são possíveis
\end{frame}

% Macros
\begin{frame}
  \frametitle{\filename{tipografia.tex}}
  \Huge
  Use macros para textos mais semânticos
\end{frame}

% Macros
\begin{frame}
  \frametitle{\filename{tipografia.tex}}
  \Huge
  Pacote \code{minted}: exemplos de código.
\end{frame}
