% artigo.tex %%%%%%%%%%%%%%
\begin{frame}[standout]
  \huge
  \filename{artigo.tex}
\end{frame}

% Dar uma olhada no arquivo. Ensinar a distinção entre o preâmbulo e o corpo do
% documento.
\begin{frame}
  \frametitle{\filename{artigo.tex}}
  \LARGE
  Exemplo de arquivo comum em \LaTeX: \filename{artigo.tex}
\end{frame}

% Explicar as opções da classe article que escolhi
\begin{frame}[fragile]
  \frametitle{\filename{artigo.tex}}
  \huge
  Classes comuns:
  \begin{multicols}{2}
    \begin{itemize}
      \item\code{article}
      \item\code{report}
      \item\code{book}
      \item\code{letter}
      \item\code{memoir}
      \item\code{beamer}
  \end{itemize}
\end{multicols}
\end{frame}

% Vejamos quais são as opções de classe mais comuns
\begin{frame}
  \frametitle{\filename{artigo.tex}}
  \LARGE
  Opções de classe comuns:
  \begin{itemize}
    \only<1>{\item \code{10pt, 11pt, 12pt}}
    \only<1>{\item \code{a4paper, a5paper, letterpaper, …}}
    \only<1>{\item \code{fleqn}}
    \only<2>{\item \code{leqno}}
    \only<2>{\item \code{titlepage, notitlepage}}
    \only<2>{\item \code{twocolumn}}
    \only<2>{\item \code{twoside, oneside}}
    \only<3>{\item \code{landscape}}
    \only<3>{\item \code{openright, openany}}
    \only<3>{\item \code{draft}}
  \end{itemize}
\end{frame}

% Compilar o documento várias vezes, com opções diferentes
\begin{frame}
  \frametitle{\filename{artigo.tex}}
  \Huge
  Testar diferentes opções de classe
\end{frame}

% Olharemos agora os pacotes carregados
\begin{frame}
  \frametitle{\filename{artigo.tex}}
  \Huge
  Pacotes: \code{polyglossia, blindtext} e \code{hyperref}
\end{frame}

% Mudar o comando author para o seguinte
\begin{frame}[fragile]
  \frametitle{\filename{artigo.tex}}
  \LARGE
  Colocar um email abaixo dessa linha:
  \begin{minted}[autogobble,fontsize=\LARGE,breaklines]{latex}
   \author{Rafael Beraldo}
  \end{minted}
\end{frame}

\begin{frame}
  \frametitle{\filename{artigo.tex}}
  \Huge
  Adicionar o pacote \code{microtype}
\end{frame}

% Mostrar as diferenças entre usar ou não a protrusão e extensão de caracteres
\begin{frame}
  \frametitle{\filename{artigo.tex}}
  \LARGE
  Manual do \code{microtype}:\\
  \url{www.ctan.org/pkg/microtype}
\end{frame}

% O corpo do documento. O que são esses dois primeiros comandos?
\begin{frame}[fragile]
  \frametitle{\filename{artigo.tex}}
  \begin{minted}[autogobble,fontsize=\LARGE,breaklines]{latex}
  \begin{document}
  \frenchspacing
  \maketitle
  …
  \end{document}
  \end{minted}
\end{frame}

% \frenchspacing era utilizado no século 19, mas não mais
\begin{frame}
  \frametitle{\filename{artigo.tex}}
  \Large
  Exemplo de \mintinline{latex}{\frenchspacing}:

  \vspace{1em}
  \begin{minipage}{.45\textwidth}
    \nonfrenchspacing
    \mintinline{latex}{\nonfrenchspacing}:

    “A poesia vogon é, como todos sabem, a terceira pior do Universo. Em segundo
    lugar vem a poesia dos azgodos de Kria.”
  \end{minipage}
  \hspace{.05\textwidth}
  \begin{minipage}{.45\textwidth}
    \frenchspacing
    \mintinline{latex}{\frenchspacing}:

    “A poesia vogon é, como todos sabem, a terceira pior do Universo. Em segundo
    lugar vem a poesia dos azgodos de Kria.”
  \end{minipage}
\end{frame}

% Como organizar seu documento em seções
\begin{frame}
  \frametitle{\filename{artigo.tex}}
  \Large
  Comandos para seccionar o documento:

  \begin{itemize}
    \only<1>{\item \code{\textbackslash part}}
    \only<1>{\item \code{\textbackslash chapter}} (apenas classes \code{book} e
    \code{report})
    \only<1>{\item \code{\textbackslash section}}
    \only<1>{\item \code{\textbackslash subsection}}
    \only<1>{\item \code{\textbackslash subsubsection}}
    \only<1>{\item \code{\textbackslash paragraph}}
    \only<1>{\item \code{\textbackslash subparagraph}}
  \end{itemize}
\end{frame}

% Aquivos auxiliares
\begin{frame}[fragile]
  \frametitle{\filename{artigo.tex}}
  \LARGE
  Arquivos auxiliares:

  \begin{minted}[autogobble,fontsize=\Large,breaklines]{bash}
    artigo-exemplo.aux
    artigo-exemplo.log
    artigo-exemplo.out
    artigo-exemplo.pdf
    artigo-exemplo.tex
  \end{minted}
\end{frame}

% Para limpar aquivos auxiliares, uma das possibilidades é utilizar o latexmk
% com a opção -c (clean)
\begin{frame}
  \frametitle{\filename{artigo.tex}}
  \Huge
  Limpar arquivos auxiliares:

  \code{\$ latexmk -c}
\end{frame}

\begin{frame}
  \frametitle{\filename{artigo.tex}}
  \Huge
  Resolver \filename{artigo-exercicio.tex}
\end{frame}
