% posicao-texto.tex %%%%%%%%%%%%%%
\begin{frame}[standout]
  \Huge
   \filename{posicao-texto.tex}
\end{frame}

% Nosso certificado pode ter ficado legal, mas poderia ser melhor ainda se
% ajustarmos o texto em relação à página.
\begin{frame}
  \frametitle{\filename{posicao-texto.tex}}
  \Huge
  Problemas com o certificado?
\end{frame}

% Antes de controlar a posição do texto, temos que entender o que é um
% ambiente.
\begin{frame}[fragile]
  \frametitle{\filename{posicao-texto.tex}}
  \Huge
  Ambientes:

  \begin{minted}[autogobble,fontsize=\Huge,breaklines]{latex}
    \begin{ambiente}
      …
    \end{ambiente}
  \end{minted}
\end{frame}

% Três ambientes para controlar posição.
\begin{frame}[fragile]
  \frametitle{\filename{posicao-texto.tex}}
  \Large
  Ambientes \code{center}, \code{flushleft} e \code{flushright}

  % Código:
  \begin{minted}[autogobble,fontsize=\Large,breaklines]{latex}
    \begin{center}
      Este texto será centralizado.
    \end{center}
  \end{minted}

  % Resultado:
  \begin{center}
    Este texto será centralizado.
  \end{center}
\end{frame}

% Ainda podemos controlar o espaço dentro de uma linha.
\begin{frame}[fragile]
  \frametitle{\filename{posicao-texto.tex}}
  \Huge
  \mintinline{latex}{\hspace{comprimento}}
\end{frame}

% Por exemplo, um espaço de 1,5cm:
\begin{frame}[fragile]
  \frametitle{\filename{posicao-texto.tex}}
  \LARGE
  \begin{minted}[autogobble,fontsize=\LARGE,breaklines]{latex}
    Essa frase\hspace{1.5cm} está esticada.
  \end{minted}
  \vspace{1em}

  Essa frase\hspace{1.5cm} está esticada.
\end{frame}

% O LaTeX aceita uma série de unidades
\begin{frame}[fragile]
  \frametitle{\filename{posicao-texto.tex}}
  \LARGE
  Unidades que o \LaTeX{} conhece:

  \begin{multicols}{2}
    \begin{itemize}
      \item\code{mm}
      \item\code{cm}
      \item\code{in}
      \item\code{pt}
      \item\code{em}
      \item\code{ex}
      \item\mintinline{latex}{\textheight}
      \item\mintinline{latex}{\textwidth}
      \item\mintinline{latex}{\pageheight}
      \item\mintinline{latex}{\pageheight}
    \end{itemize}
  \end{multicols}
\end{frame}

% O comando \hfill preenche todo o espaço disponível
\begin{frame}[fragile]
  \frametitle{\filename{posicao-texto.tex}}
  \LARGE
  \mintinline{latex}{Começo\hfill meio\hfill fim}
  \vspace{1em}

  Começo\hfill meio\hfill fim
\end{frame}

% E, finalmente, existem os comandos \vspace e \hfill
\begin{frame}[fragile]
  \frametitle{\filename{posicao-texto.tex}}
  \huge
  Comandos análogos:

  \mintinline{latex}{\vspace{comprimento}}
  \vspace{1em}

  \mintinline{latex}{\vfill}
\end{frame}

% Exercício: terminar o certificado que começamos antes
\begin{frame}
  \frametitle{\filename{posicao-texto.tex}}
  \Huge
  Resolver \filename{posicao-texto-exercicio.tex}
\end{frame}

% Sugestão para o visual final do certificado
\begin{frame}[plain]
  \begin{center}
    {\huge\textbf{OpenSanca}}\\[2em]
    {\LARGE\textsc{Certificado}}
  \end{center}

    \noindent Certificamos que Tal Pessoa da Silva participou de um curso em
    nosso grupo no dia 28 de maio de 2016 e está qualificado para editar textos
    em \LaTeX.

    \vfill
    \begin{flushright}
      \emph{Os Organizadores}\\
      \emph{OpenSanca}\\
    \end{flushright}
\end{frame}
