% abntex2-example.bib %%%%%%%%%%%%%%
\begin{frame}[standout]
  \Huge
  \filename{abntex2-example.bib}
\end{frame}

% O BibTeX tem dois arquivos principais
\begin{frame}
  \frametitle{\filename{abntex2-example.bib}}
  \Huge
  \hologo{BibTeX}: database (\code{bib}) e estilo (\code{bst})
\end{frame}

% Um exemplo de entrada bibliográfica
\begin{frame}[fragile]
  \frametitle{\filename{abntex2-example.bib}}
  \Huge
  Arquivo \code{.bib}:

  \begin{minted}[autogobble,fontsize=\large,breaklines]{latex}
    @article{greenwade93,
      author  = "George D. Greenwade",
      title   = "The {C}omprehensive {T}ex {A}rchive {N}etwork ({CTAN})",
      year    = "1993",
      journal = "TUGBoat",
      volume  = "14",
      number  = "3",
      pages   = "342--351"
    }
  \end{minted}
\end{frame}

% No local desejado, colocamos a bibliografia
\begin{frame}[fragile]
  \frametitle{\filename{abntex2-example.bib}}
  \LARGE
  \verb+\bibliography{arquivo}+
\end{frame}

% Para citar, basta usar um desses comandos
\begin{frame}[fragile]
  \frametitle{\filename{abntex2-example.bib}}
  \LARGE
  \verb+\cite[p.~20]{greenwade93}+

  \verb+\citeonline[p.~20]{greenwade93}+
\end{frame}

% Mais um live coding!
\begin{frame}
  \frametitle{\filename{abntex2-example.bib}}
  \Huge
  Exemplo: \filename{abntex2-example.bib}
\end{frame}
