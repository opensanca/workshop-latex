% poliglota-exercicio.tex %%%%%%%%%%%%%%
\begin{frame}[standout]
  \huge
  \filename{poliglota-exercicio.tex}
\end{frame}

% No exemplo anterior, espaco-branco.tex, os acentos — na verdade, diacríticos
% — não apareceram. Alguém saberia o motivo?
\begin{frame}
  \frametitle{\filename{poliglota-exercicio.tex}}
  \huge
  Acentos não apareciam em \filename{espaco-branco.tex}
\end{frame}

% Para resolver, teremos que usar pacotes
\begin{frame}
  \frametitle{\filename{poliglota-exercicio.tex}}
  \Huge
  Solução: pacotes
\end{frame}

% Para carregar pacotes, usamos esta sintaxe
\begin{frame}[fragile]
  \frametitle{\filename{poliglota-exercicio.tex}}
  \begin{minted}[autogobble,fontsize=\LARGE,breaklines]{latex}
    \usepackage[opções]{pacote}
  \end{minted}
\end{frame}

% Para resolvermos a falta de diacríticos, usaremos o pacote polyglossia
\begin{frame}
  \frametitle{\filename{poliglota-exercicio.tex}}
  \Huge
  Pacote \code{polyglossia}
\end{frame}

\begin{frame}
  \frametitle{\filename{poliglota-exercicio.tex}}
  \Huge
  O \code{polyglossia} traz benefícios como:
  \begin{itemize}
    \only<1>{\item Hifenização}
    \only<2>{\item Strings como \mintinline{latex}{\today}}
    \only<3>{\item Convenções tipográficas localizadas}
  \end{itemize}
\end{frame}

% Ao exercício.
%
% Sabemos para que serve o polyglossia. Perguntar como ele seria implementado,
% e onde seria colocado no código.
%
% Também estudar a sintaxe para selecionar línguas. Explicar o motivo pelo qual
% a opção de pacote [brazil] não é mais usada: fica difícil selecionar várias
% línguas e suas opções.
\begin{frame}
  \frametitle{\filename{poliglota-exercicio.tex}}
  \Huge
  Como carregar o pacote \code{polyglossia}?
\end{frame}

\begin{frame}[fragile]
  \frametitle{\filename{poliglota-exercicio.tex}}
  \begin{minted}[autogobble,fontsize=\Large,breaklines]{latex}
    \usepackage{polyglossia}
    \setdefaultlanguage{brazil}
  \end{minted}
\end{frame}

% Para encontrar ajuda, podemos ler a documentação oficial dos pacotes que
% estamos usando. Mostrar outras opções do polyglossia, por exemplo.
\begin{frame}
  \frametitle{\filename{poliglota-exercicio.tex}}
  \Huge
  Comprehensive \TeX{} Archive Network

  \url{ctan.org}
\end{frame}

\begin{frame}
  \frametitle{\filename{poliglota-exercicio.tex}}
  \huge
  \url{https://www.ctan.org/pkg/polyglossia}
\end{frame}
