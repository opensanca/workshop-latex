% tabelas.tex %%%%%%%%%%%%%%
\begin{frame}[standout]
  \Huge
  \filename{tabelas.tex}
\end{frame}

\begin{frame}[fragile]
  \frametitle{\filename{tabelas.tex}}
  \Huge
  A abordagem é diferentes dos programas WISIWYG.
\end{frame}

% Exemplo simples de tabela com o ambiente tabular
\begin{frame}[fragile]
  \frametitle{\filename{tabelas.tex}}
  \Large
  Exemplo do ambiente \code{tabular}:
  \vspace{1em}

  \begin{minipage}{.65\textwidth}
    \begin{minted}[autogobble,fontsize=\large,breaklines]{latex}
      \begin{tabular}{lcr}
        1 & 2 & 3\\
        4 & 5 & 6\\
        7 & 8 & 9
      \end{tabular}
    \end{minted}
  \end{minipage}
  \hspace{.05\textwidth}
  \begin{minipage}{.25\textwidth}
    \begin{tabular}{lcr}
      1 & 2 & 3\\
      4 & 5 & 6\\
      7 & 8 & 9
    \end{tabular}
  \end{minipage}
\end{frame}

% Com mais alguns detalhes como linhas verticais e horizontais
\begin{frame}[fragile]
  \frametitle{\filename{tabelas.tex}}
  \Large
  Linhas horizontais e verticais:
  \vspace{1em}

  \begin{minipage}{.65\textwidth}
    \begin{minted}[autogobble,fontsize=\large,breaklines]{latex}
      \begin{tabular}{l|c|r}
        \hline
        1 & 2 & 3\\
        4 & 5 & 6\\
        7 & 8 & 9\\
        \hline
      \end{tabular}
    \end{minted}
  \end{minipage}
  \hspace{.05\textwidth}
  \begin{minipage}{.25\textwidth}
    \begin{tabular}{l|c|r}
      \hline
      1 & 2 & 3\\
      4 & 5 & 6\\
      7 & 8 & 9\\
      \hline
    \end{tabular}
  \end{minipage}
\end{frame}

% O espaço em branco deixa o texto respirar e evita a tirania das linhas retas.
% Essa citação do Bringhurst é fantástica.
\begin{frame}
  \frametitle{\filename{tabelas.tex}}
  \large
  \begin{quote}
    Assim como o texto, as tabelas ficam canhestras quando abordadas de forma
    puramente técnica. Boas soluções tipográficas não costumam surgir em resposta
    a perguntas do tipo “Como posso enfiar essa quantidade de caracteres naquele
    tanto de espaço?”.\\\hfill (Robert Bringhurst, \emph{Elementos do Estilo
    Tipográfico)}
  \end{quote}
\end{frame}

% Vejamos alguns exemplos de tabela em tabelas.tex.
\begin{frame}
  \frametitle{\filename{tabelas.tex}}
  \Huge
  Vejamos \filename{tabelas.tex}
\end{frame}

% Vamos revisar o que aprendemos no exemplo.
\begin{frame}[fragile]
  \frametitle{\filename{tabelas.tex}}
  \Huge
  Aprendemos:

  \begin{itemize}
    \only<1>{\item \code{tabular}}
    \only<1>{\item tipografia da tabela}
    \only<2>{\item quebras de linhas}
    \only<2>{\item \code{booktabs}}
    \only<3>{\item \mintinline{latex}{\multicolumn}}
    \only<3>{\item \code{longtable}}
  \end{itemize}
\end{frame}

% Exercício: formatar uma lista em CSV
\begin{frame}
  \frametitle{\filename{tabelas.tex}}
  \Huge
  Resolver: \filename{tabelas-exercicio.tex}
\end{frame}

% Temos declarado exatamente onde desejamos que a tabela fique, mas essa
% abordagem nem sempre é boa, porque interrompe o fluxo do texto. Vamos
% aprender mais sobre os ambientes do tipo float.
\begin{frame}
  \frametitle{\filename{tabelas.tex}}
  \Huge
  \only<1>{Ambiente \code{tabular} coloca a tabela após o texto}
  \only<2>{Padrão profissional: \emph{floats}}
  \only<3>{Dois floats: \code{table} e \code{figure}}
\end{frame}

% A sintaxe do ambiente table
\begin{frame}[fragile]
  \frametitle{\filename{tabelas.tex}}
  \huge
  Sintaxe de \code{table}:
  \vspace{1em}

    \begin{minted}[autogobble,fontsize=\huge,breaklines]{latex}
      \begin{table}[posição]
        …
      \end{table}
    \end{minted}
\end{frame}

% Exemplo, com os comandos \centering, \caption e \label.
\begin{frame}[fragile]
  \frametitle{\filename{tabelas.tex}}
  \Large
  Veja a tabela~\ref{tab:numerosUmNove}:

  \begin{minipage}{.65\textwidth}
    \begin{minted}[autogobble,fontsize=\large,breaklines]{latex}
    \begin{table}
      \centering
      \begin{tabular}{lcr}
      1 & 2 & 3\\
      4 & 5 & 6\\
      7 & 8 & 9
      \end{tabular}
      \caption{Números de 1 a 9}
      \label{tab:numerosUmNove}
    \end{table}
    \end{minted}
  \end{minipage}
  \hspace{.05\textwidth}
  \begin{minipage}{.25\textwidth}
    \begin{table}
      \centering
      \begin{tabular}{lcr}
        1 & 2 & 3\\
        4 & 5 & 6\\
        7 & 8 & 9
      \end{tabular}
      \caption{Números de 1 a 9}
      \label{tab:numerosUmNove}
    \end{table}
  \end{minipage}
\end{frame}

% Voltaremos ao arquivo anterior para brincar com esses conceitos e
% referências cruzadas.
\begin{frame}
  \frametitle{\filename{tabelas.tex}}
  \Huge
  Voltemos a \filename{tabelas.tex}
\end{frame}

% Exercício: uma tabela do zero, usando o ambiente table e referenciando a
% tabela num parágrafo anterior.
\begin{frame}
  \frametitle{\filename{tabelas.tex}}
  \Huge
  Resolver: \filename{tabelas-questionario-exercicio.tex}
\end{frame}

\begin{frame}
  \frametitle{\filename{tabelas.tex}}
  \Huge
  Pacote \code{tabularx}
\end{frame}

\begin{frame}
  \frametitle{\filename{tabelas.tex}}
  \Huge
  Veja o documento \filename{fala.md} para mais ferramentas
\end{frame}
