\documentclass[final]{beamer}
\usepackage{hyperref,xspace,graphicx,microtype,minted}
\usepackage[brazil]{babel}

\usetheme[sectionpage=simple,numbering=none]{metropolis}
\beamertemplatenavigationsymbolsempty

\newcommand{\filename}[1]{\texttt{#1}}
\newcommand{\code}[1]{\texttt{#1}}

\title{Olá, \LaTeX!}
\author{Rafael Beraldo}
\date{28 de maio de 2016}

\begin{document}
\maketitle

\begin{frame}[standout]
  \Huge História e filosofia vão aqui
\end{frame}

% Considerações iniciais
\begin{frame}
  \only<1>{\Huge\LaTeX{} é uma linguagem de marcação de texto}
  \only<2>{\Huge Você \emph{declara} o documento}
  \only<3>{\Huge É como um tipógrafo profissional à sua disposição}
  \only<4>{\Huge Comandos são semânticos}
\end{frame}

% Sintaxe dos comandos
\begin{frame}[fragile]
  \begin{minted}[autogobble,fontsize=\huge,breaklines]{latex}
    \section{Introdução}
  \end{minted}
\end{frame}

% Exercício: hello-world.tex %%%%%%%%%%%%%
\begin{frame}[standout]
  \huge
  Exercício: \filename{hello-world.tex}
\end{frame}

\begin{frame}[fragile]
  \frametitle{\filename{hello-world.tex}}
  \begin{minted}[autogobble,fontsize=\Large,breaklines]{latex}
    \documentclass{article}
    \begin{document}
      Hello, world!
    \end{document}
  \end{minted}
\end{frame}

\begin{frame}[fragile]
  \frametitle{\filename{hello-world.tex}}
  \begin{minted}[autogobble,fontsize=\LARGE,breaklines]{bash}
    lualatex hello-world.tex
  \end{minted}
\end{frame}

\begin{frame}[fragile]
  \frametitle{\filename{hello-world.tex}}
  \begin{minted}[autogobble,breaklines]{bash}
    % hello-world.tex
    %
    % Rafael Beraldo <rberaldo@cabaladada.org>
    % Workshop de LaTeX do Opensanca
    % 28 de maio de 2016
  \end{minted}
\end{frame}

% Exercício: espaco-branco.tex %%%%%%%%%%%%%%
% No exercício, temos dois parágrafos do Guia do Mochileiro das Galáxias.
\begin{frame}[standout]
  \huge
  Exercício: \filename{espaco-branco.tex}
\end{frame}

% Ensinar que, para criar uma nova linha, usamos estes comandos.
\begin{frame}[fragile]
  \frametitle{\filename{espaco-branco.tex}}
  \begin{minted}[autogobble,fontsize=\LARGE,breaklines]{latex}
    \\newline

    \\
  \end{minted}
\end{frame}

% Mostrar a extensão dos exercícios; pedir que o resolvam. Mostrar como os
% comentários contém as instruções.
\begin{frame}
  \frametitle{\filename{espaco-branco-exercicio.tex}}
  \LARGE
  Vamos resolver \filename{espaco-branco-exercicio.tex}
\end{frame}

% Exercício: poliglota-exercicio.tex %%%%%%%%%%%%%%
\begin{frame}[standout]
  \huge
  Exercício: \filename{poliglota-exercicio.tex}
\end{frame}

% No exemplo anterior, espaco-branco.tex, os acentos --- na verdade,
% diacríticos --- não apareceram. Alguém saberia o motivo?
\begin{frame}
  \frametitle{\filename{poliglota-exercicio.tex}}
  \LARGE
  Acentos não apareciam em \filename{espaco-branco.tex}
\end{frame}

% Para resolver, teremos que usar pacotes
\begin{frame}
  \frametitle{\filename{poliglota-exercicio.tex}}
  \Huge
  Solução: pacotes
\end{frame}

% Para carregar pacotes, usamos esta sintaxe
\begin{frame}[fragile]
  \frametitle{\filename{poliglota-exercicio.tex}}
  \begin{minted}[autogobble,fontsize=\LARGE,breaklines]{latex}
    \usepackage[opções]{pacote}
  \end{minted}
\end{frame}

% Para resolvermos a falta de diacríticos, usaremos o pacote polyglossia
\begin{frame}
  \frametitle{\filename{poliglota-exercicio.tex}}
  \Huge
  Pacote \code{polyglossia}
\end{frame}

\begin{frame}
  \frametitle{\filename{poliglota-exercicio.tex}}
  \Huge
  O \code{polyglossia} traz benefícios como:
  \begin{itemize}
    \only<1>{\item Hifenização}
    \only<2>{\item Strings como \code{\textbackslash today}}
    \only<3>{\item Convenções tipográficas localizadas}
  \end{itemize}
\end{frame}

% Vamos ao exercício.
%
% Sabemos para que serve o polyglossia. Perguntar como ele seria implementado,
% e onde seria colocado no código.
%
% Também estudar a sintaxe para selecionar línguas. Explicar o motivo pelo qual
% a opção de pacote [brazil] não é mais usada: fica difícil selecionar várias
% línguas e suas opções.
\begin{frame}
  \frametitle{\filename{poliglota-exercicio.tex}}
  \Huge
  Como carregar o pacote \code{polyglossia}?
\end{frame}

\begin{frame}[fragile]
  \frametitle{\filename{poliglota-exercicio.tex}}
  \begin{minted}[autogobble,fontsize=\LARGE,breaklines]{latex}
    \usepackage{polyglossia}
    \setdefaultlanguage{brazil}
  \end{minted}
\end{frame}

% Para encontrar ajuda, podemos ler a documentação oficial dos pacotes que
% estamos usando. Mostrar outras opções do polyglossia, por exemplo.
\begin{frame}
  \frametitle{\filename{poliglota-exercicio.tex}}
  \Huge
  Comprehensive \TeX{} Archive Network

  \url{ctan.org}
\end{frame}

\begin{frame}
  \frametitle{\filename{poliglota-exercicio.tex}}
  \huge
  \url{https://www.ctan.org/pkg/polyglossia}
\end{frame}

% Exercício: artigo-exercicio.tex %%%%%%%%%%%%%%
\begin{frame}[standout]
  \huge
  Exercício: \filename{artigo-exercicio.tex}
\end{frame}

\begin{frame}
  \frametitle{\filename{artigo-exercicio.tex}}
  \Huge
  Exemplo de arquivo comum em \LaTeX: \filename{artigo-exemplo.tex}
\end{frame}

\end{document}
