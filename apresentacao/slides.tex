\documentclass[draft]{beamer}
  \frametitle{\filename{poliglota-exercicio.tex}}
\usepackage{hyperref,xspace,graphicx,microtype,minted,multicol}
\usepackage[brazil]{babel}

\usetheme[sectionpage=simple,numbering=none]{metropolis}
\beamertemplatenavigationsymbolsempty

\newcommand{\filename}[1]{\texttt{#1}}
\newcommand{\code}[1]{\texttt{#1}}

\title{Olá, \LaTeX!}
\author{Rafael Beraldo}
\date{28 de maio de 2016}

\begin{document}
\maketitle

\begin{frame}[standout]
  \Huge História e filosofia vão aqui
\end{frame}

% Considerações iniciais
\begin{frame}
  \only<1>{\Huge\LaTeX{} é uma linguagem de marcação de texto}
  \only<2>{\Huge Você \emph{declara} o documento}
  \only<3>{\Huge É como um tipógrafo profissional à sua disposição}
  \only<4>{\Huge Comandos são semânticos}
\end{frame}

% Sintaxe dos comandos
\begin{frame}[fragile]
  \begin{minted}[autogobble,fontsize=\huge,breaklines]{latex}
    \section{Introdução}
  \end{minted}
\end{frame}

% Exercício: hello-world.tex %%%%%%%%%%%%%
\begin{frame}[standout]
  \huge
  Exercício: \filename{hello-world.tex}
\end{frame}

\begin{frame}[fragile]
  \frametitle{\filename{hello-world.tex}}
  \begin{minted}[autogobble,fontsize=\Large,breaklines]{latex}
    \documentclass{article}
    \begin{document}
      Hello, world!
    \end{document}
  \end{minted}
\end{frame}

\begin{frame}[fragile]
  \frametitle{\filename{hello-world.tex}}
  \begin{minted}[autogobble,fontsize=\LARGE,breaklines]{bash}
    lualatex hello-world.tex
  \end{minted}
\end{frame}

\begin{frame}[fragile]
  \frametitle{\filename{hello-world.tex}}
  \begin{minted}[autogobble,breaklines]{latex}
    % hello-world.tex
    %
    % Rafael Beraldo <rberaldo@cabaladada.org>
    % Workshop de LaTeX do Opensanca
    % 28 de maio de 2016
  \end{minted}
\end{frame}

% Exercício: espaco-branco.tex %%%%%%%%%%%%%%
% No exercício, temos dois parágrafos do Guia do Mochileiro das Galáxias.
\begin{frame}[standout]
  \huge
  Exercício: \filename{espaco-branco.tex}
\end{frame}

% Ensinar que, para criar uma nova linha, usamos estes comandos.
\begin{frame}[fragile]
  \frametitle{\filename{espaco-branco.tex}}
  \begin{minted}[autogobble,fontsize=\LARGE,breaklines]{latex}
    \\newline

    \\
  \end{minted}
\end{frame}

% Mostrar a extensão dos exercícios; pedir que o resolvam. Mostrar como os
% comentários contém as instruções.
\begin{frame}
  \frametitle{\filename{espaco-branco-exercicio.tex}}
  \LARGE
  Resolver \filename{espaco-branco-exercicio.tex}
\end{frame}

% Exercício: poliglota-exercicio.tex %%%%%%%%%%%%%%
\begin{frame}[standout]
  \huge
  Exercício: \filename{poliglota-exercicio.tex}
\end{frame}

% No exemplo anterior, espaco-branco.tex, os acentos --- na verdade,
% diacríticos --- não apareceram. Alguém saberia o motivo?
\begin{frame}
  \frametitle{\filename{poliglota-exercicio.tex}}
  \LARGE
  Acentos não apareciam em \filename{espaco-branco.tex}
\end{frame}

% Para resolver, teremos que usar pacotes
\begin{frame}
  \frametitle{\filename{poliglota-exercicio.tex}}
  \Huge
  Solução: pacotes
\end{frame}

% Para carregar pacotes, usamos esta sintaxe
\begin{frame}[fragile]
  \frametitle{\filename{poliglota-exercicio.tex}}
  \begin{minted}[autogobble,fontsize=\LARGE,breaklines]{latex}
    \usepackage[opções]{pacote}
  \end{minted}
\end{frame}

% Para resolvermos a falta de diacríticos, usaremos o pacote polyglossia
\begin{frame}
  \frametitle{\filename{poliglota-exercicio.tex}}
  \Huge
  Pacote \code{polyglossia}
\end{frame}

\begin{frame}
  \frametitle{\filename{poliglota-exercicio.tex}}
  \Huge
  O \code{polyglossia} traz benefícios como:
  \begin{itemize}
    \only<1>{\item Hifenização}
    \only<2>{\item Strings como \code{\textbackslash today}}
    \only<3>{\item Convenções tipográficas localizadas}
  \end{itemize}
\end{frame}

% Ao exercício.
%
% Sabemos para que serve o polyglossia. Perguntar como ele seria implementado,
% e onde seria colocado no código.
%
% Também estudar a sintaxe para selecionar línguas. Explicar o motivo pelo qual
% a opção de pacote [brazil] não é mais usada: fica difícil selecionar várias
% línguas e suas opções.
\begin{frame}
  \frametitle{\filename{poliglota-exercicio.tex}}
  \Huge
  Como carregar o pacote \code{polyglossia}?
\end{frame}

\begin{frame}[fragile]
  \frametitle{\filename{poliglota-exercicio.tex}}
  \begin{minted}[autogobble,fontsize=\Large,breaklines]{latex}
    \usepackage{polyglossia}
    \setdefaultlanguage{brazil}
  \end{minted}
\end{frame}

% Para encontrar ajuda, podemos ler a documentação oficial dos pacotes que
% estamos usando. Mostrar outras opções do polyglossia, por exemplo.
\begin{frame}
  \frametitle{\filename{poliglota-exercicio.tex}}
  \Huge
  Comprehensive \TeX{} Archive Network

  \url{ctan.org}
\end{frame}

\begin{frame}
  \frametitle{\filename{poliglota-exercicio.tex}}
  \huge
  \url{https://www.ctan.org/pkg/polyglossia}
\end{frame}

% Exercício: artigo-exercicio.tex %%%%%%%%%%%%%%
\begin{frame}[standout]
  \huge
  Exercício: \filename{artigo-exercicio.tex}
\end{frame}

% Dar uma olhada no arquivo. Ensinar a distinção entre o preâmbulo e o corpo do
% documento.
\begin{frame}
  \frametitle{\filename{artigo-exercicio.tex}}
  \LARGE
  Exemplo de arquivo comum em \LaTeX: \filename{artigo-exemplo.tex}
\end{frame}

% Explicar as opções da classe article que escolhi
\begin{frame}[fragile]
  \frametitle{\filename{artigo-exercicio.tex}}
  \LARGE
  Classes comuns:
  \begin{multicols}{2}
    \begin{itemize}
      \only<1>{\item \code{article}}
      \only<1>{\item \code{report}}
      \only<1>{\item \code{book}}
      \only<1>{\item \code{letter}}
      \only<1>{\item \code{memoir}}
      \only<1>{\item \code{beamer}}
  \end{itemize}
\end{multicols}
\end{frame}

% Vejamos quais são as opções de classe mais comuns
\begin{frame}
  \frametitle{\filename{poliglota-exercicio.tex}}
  \LARGE
  Opções de classe comuns:
  \begin{itemize}
    \only<1>{\item \code{10pt, 11pt, 12pt}}
    \only<1>{\item \code{a4paper, a5paper, letterpaper, …}}
    \only<1>{\item \code{fleqn}}
    \only<2>{\item \code{leqno}}
    \only<2>{\item \code{titlepage, notitlepage}}
    \only<2>{\item \code{twocolumn}}
    \only<2>{\item \code{twoside, oneside}}
    \only<3>{\item \code{landscape}}
    \only<3>{\item \code{openright, openany}}
    \only<3>{\item \code{draft}}
  \end{itemize}
\end{frame}

% Compilar o documento várias vezes, com opções diferentes
\begin{frame}
  \frametitle{\filename{poliglota-exercicio.tex}}
  \huge
  Vamos testar algumas opções?
\end{frame}

% Olharemos agora os pacotes carregados
\begin{frame}
  \frametitle{\filename{poliglota-exercicio.tex}}
  \huge
  Pacotes: \code{polyglossia, blindtext} e \code{hyperref}
\end{frame}

% Mudar o comando author para o seguinte
\begin{frame}[fragile]
  \frametitle{\filename{poliglota-exercicio.tex}}
  \LARGE
  Colocar um email abaixo dessa linha:
  \begin{minted}[autogobble,fontsize=\LARGE,breaklines]{latex}
   \author{Rafael Beraldo}
  \end{minted}
\end{frame}

\begin{frame}
  \frametitle{\filename{poliglota-exercicio.tex}}
  \Huge
  Adicionar o pacote \code{microtype}
\end{frame}

% Mostrar as diferenças entre usar ou não a protrusão e extensão de caracteres
\begin{frame}
  \frametitle{\filename{poliglota-exercicio.tex}}
  \LARGE
  Manual do \code{microtype}:\\
  \url{www.ctan.org/pkg/microtype}
\end{frame}

% O corpo do documento. O que são esses dois primeiros comandos?
\begin{frame}[fragile]
  \frametitle{\filename{poliglota-exercicio.tex}}
  \begin{minted}[autogobble,fontsize=\LARGE,breaklines]{latex}
  \begin{document}
  \frenchspacing
  \maketitle
  …
  \end{document}
  \end{minted}
\end{frame}

% \frenchspacing era utilizado no século 19, mas não mais
\begin{frame}
  \frametitle{\filename{poliglota-exercicio.tex}}
  \Large
  Exemplo de \code{\textbackslash frenchspacing}:

  \vspace{1em}
  \begin{minipage}{.45\textwidth}
    \nonfrenchspacing
    “A poesia vogon é, como todos sabem, a terceira pior do Universo. Em segundo
    lugar vem a poesia dos azgodos de Kria.”
  \end{minipage}
  \hspace{.05\textwidth}
  \begin{minipage}{.45\textwidth}
    \frenchspacing
    “A poesia vogon é, como todos sabem, a terceira pior do Universo. Em segundo
    lugar vem a poesia dos azgodos de Kria.”
  \end{minipage}
\end{frame}

% Como organizar seu documento em seções
\begin{frame}
  \frametitle{\filename{poliglota-exercicio.tex}}
  \Large
  Comandos para seccionar o documento:

  \begin{itemize}
    \only<1>{\item \code{\textbackslash part}}
    \only<1>{\item \code{\textbackslash chapter}} (apenas classes \code{book} e
    \code{report})
    \only<1>{\item \code{\textbackslash section}}
    \only<1>{\item \code{\textbackslash subsection}}
    \only<1>{\item \code{\textbackslash subsubsection}}
    \only<1>{\item \code{\textbackslash paragraph}}
    \only<1>{\item \code{\textbackslash subparagraph}}
  \end{itemize}
\end{frame}

% Aquivos auxiliares
\begin{frame}[fragile]
  \frametitle{\filename{poliglota-exercicio.tex}}
  \Large
  Arquivos auxiliares:

  \begin{minted}[autogobble,fontsize=\Large,breaklines]{bash}
    artigo-exemplo.aux
    artigo-exemplo.log
    artigo-exemplo.out
    artigo-exemplo.pdf
    artigo-exemplo.tex
  \end{minted}
\end{frame}

% Para limpar aquivos auxiliares, uma das possibilidades é utilizar o latexmk
% com a opção -c (clean)
\begin{frame}
  \frametitle{\filename{poliglota-exercicio.tex}}
  \Huge
  Para limpar os arquivos auxiliares:

  \code{\$ latexmk -c}
\end{frame}

\begin{frame}
  \frametitle{\filename{poliglota-exercicio.tex}}
  \Huge
  Resolver \filename{poliglota-exercicio.tex}
\end{frame}

% Exercício: ?.tex %%%%%%%%%%%%%%
\begin{frame}[standout]
  \huge
  Exercício: \filename{?.tex}
\end{frame}

\end{document}
