% layouts-pagina.tex %%%%%%%%%%%%%%
\begin{frame}[standout]
  \Huge
   \filename{layouts-pagina.tex}
\end{frame}

% Vamos mudar a opção de classe de onecolumn para twocolumn e carregar o pacote
% showframe.
\begin{frame}
  \frametitle{\filename{layouts-pagina.tex}}
  \huge
  \only<1>{Copiar solução de \filename{fontes-exercio.tex} em
  \filename{layouts-pagina.tex}}
  \only<2>{Mudar para \code{twocolumn}, carregar o pacote \code{showframe}}
\end{frame}

% Na folha A4, apenas uma coluna coluna de texto é difícil de ler com margens
% curtas. Mas usar margens grandes desperdiça papel.
\begin{frame}
  \frametitle{\filename{layouts-pagina.tex}}
  \huge
  \code{onecolumn}: margens grandes demais

  \code{twocolumn}: nem sempre podemos
\end{frame}

% Soluções para o problema do tamanho da coluna de texto vs margens
\begin{frame}
  \frametitle{\filename{layouts-pagina.tex}}
  \huge
  Soluções:
  \begin{itemize}
    \only<1>{\item Colunas}
    \only<2>{\item \code{fullpage}}
    \only<3>{\item \code{fullpage} e entrelinhas maiores}
  \end{itemize}
\end{frame}

% Se decidirmos usar o pacote fullpage, é uma boa ideia aumentar o espaçamento
% entre as linhas.
\begin{frame}
  \frametitle{\filename{layouts-pagina.tex}}
  \huge
  Pacote \code{setspace}:

  \begin{itemize}
    \item \code{\textbackslash singlespacing}
    \item \code{\textbackslash onehalfspacing}
    \item \code{\textbackslash doublespacing}
  \end{itemize}
\end{frame}

% Outro fator que influencia o layout da página é seu estilo. Estes são os três
% comandos básicos e estilos que podemos escolher.
\begin{frame}
  \frametitle{\filename{layouts-pagina.tex}}
  \huge
  \code{\textbackslash pagestyle} e \code{\textbackslash thispagestyle}

  \begin{itemize}
    \item \code{empty}
    \item \code{plain}
    \item \code{headings}
  \end{itemize}
\end{frame}

% Outro fator que influencia o layout da página é seu estilo. Estes são os três
% comandos básicos e estilos que podemos escolher.
\begin{frame}
  \frametitle{\filename{layouts-pagina.tex}}
  \huge
  Demonstração em \filename{layouts-pagina.tex}
\end{frame}

% Exercício: faremos um certificado de conclusão do curso
\begin{frame}
  \frametitle{\filename{layouts-pagina-exercicio.tex}}
  \Huge
  Certificado de conclusão
\end{frame}

% Uma certificado incompleto
\begin{frame}[plain]

  {\huge\textbf{OpenSanca}}\\[2em]
  {\LARGE\textsc{Certificado}}

  \noindent Certificamos que Tal Pessoa da Silva participou de um curso em
  nosso grupo no dia 28 de maio de 2016 e está qualificado para editar textos
  em \LaTeX.

  \vfill
  \emph{Os Organizadores}\\
  \emph{OpenSanca}

\end{frame}

% Começar nosso certificado de conclusão de curso
\begin{frame}
  \frametitle{\filename{layouts-pagina.tex}}
  \Huge
  Resolver \filename{layouts-pagina-exercicio.tex}
\end{frame}
